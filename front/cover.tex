% !Mode:: "TeX:UTF-8"

\hitsetup{
  %******************************
  % 注意:
  %   1. 配置里面不要出现空行
  %   2. 不需要的配置信息可以删除
  %******************************
  %
  %=====
  % 秘级
  %=====
  statesecrets={公开},
  natclassifiedindex={TP309.7}, % 根据研究内容修改
  intclassifiedindex={004.6}, % 根据研究内容修改
  %
  %=========
  % 中文信息
  %=========
  ctitlecover={论文标题},%放在封面中使用,自由断行
  ctitle={论文标题},%放在原创性声明中使用
  csubtitle={一条副标题}, %一般情况没有,可以注释掉
  cxueke={工程},
  csubject={计算机科学与技术},
  caffil={哈尔滨工业大学(深圳)},
  cauthor={张三},
  csupervisor={李四 教授},
%  cassosupervisor={某某某教授}, % 副指导老师
%  ccosupervisor={某某某教授}, % 联合指导老师
  % 日期自动使用当前时间,若需指定按如下方式修改:
  cdate={2022年11月},
  cstudenttype={学术学位论文}, %非全日制教育申请学位者
  %
  %
  %=========
  % 英文信息
  %=========
  etitle={Title},
  esubtitle={This is the sub title},
  exueke={Master of Engineering},
  esubject={Electronic and Information Engineering},
  eaffil={Harbin Institute of Technology, Shenzhen},
  eauthor={San Zhang},
  esupervisor={Prof. Li Si},
%  eassosupervisor={XXX},
  edate={November, 2022},
  % 关键词用“英文逗号”分割
  ckeywords={关键词1, 关键词2, 关键词3},
  ekeywords={key word1, key word2, key word3},
}

\begin{cabstract}
  气体静压轴承由于具有运动精度高、摩擦损耗小、发热变形小、寿命长、无污染等特点,在航空航天工业、半导体工业、纺织工业和测量仪器中得到广泛应用。本文在分析国内外气体静压轴承的基础上,以改善气体静压轴承的静态特性和稳定性为目的,通过理论分析、仿真计算和实验研究对局部多孔质气体静压止推轴承进行了研究,同时分析轴承的结构参数和工作参数对局部多孔质气体静压止推轴承工作特性的影响,为局部多孔质气体静压轴承的设计和工程应用奠定理论基础。 
  
建立基于分形几何理论的多孔质石墨渗透率与分形维数之间关系的数学模型,该模型可预测多孔质石墨的渗透率,并可直观描述孔隙的大小对渗透率的影响。 

本文在理论分析的基础上,建立局部多孔质气体静压止推轴承静态特性的数学模型,在此基础上,通过工程方法和有限元方法对所建立的模型进行求解。在采用有限元方法时,首先通过加权余量法,将二阶偏微分方程降为一阶,从而,降低了对插值函数连续度的要求,同时,方便采用有限元方法进行求解。
……

\end{cabstract}

\begin{eabstract}
  Externally pressurized gas bearing has been widely used in the field of aviation, semiconductor, weave, and measurement apparatus because of its advantage of high accuracy, little friction, low heat distortion, long life-span, and no pollution. In this thesis, based on the domestic and overseas researching development about externally pressurized gas bearing, the author investigated the partial porous externally pressurized gas thrust bearing by theoretical analysis, computer simulation, and experiments to improve its static charaterictics and stability. The effects of structure and operating parameters on partial porous externally pressurized gas bearing has been studied. Therefore, a theoretical foundation for the designing and application for the partial porous externally pressurized gas bearing has been presented. 
  
Based on the fractal theory, a model was established to demonstrate the relationship between the porous graphite permeability and the fractal dimension. It can predict the permeability of porous graphite and show the effects of the pore size on the permeability. 

In this thesis, the author established a model about the static characteristics of partial porous externally pressurized gas thrust bearing, and it was analyzed by engineering solution and Finite Element Method (FEM). While using FEM, the second-order partial differential equation was reduced to one-order by adopting Galerkin weighted residual method, for decreasing the continuity degree requirement of the interpolation function and facilitating to the calculation.

…

\end{eabstract}
